\documentclass{article}
\usepackage[utf8]{inputenc}
\usepackage{amsmath}
\usepackage{amsfonts}
\usepackage{amssymb}
\usepackage{amsthm}
\usepackage{color}


\begin{document}

Lists \( \textbf{F} = (F_1, \ldots, F_n) \) of orthogonality functions are examples of \emph{"orthogonality fans"} (cf. Definition \textcolor{blue}{3.8}), and there is a notion for when a chain \( \sigma = [y_0 < \cdots < y_r] \) is orthogonal to a fan \( \textbf{F} \), written \( \sigma \perp \textbf{F} \) (cf. Definition \textcolor{blue}{3.9}). Using discrete Morse theory (cf. [\textcolor{blue}{For98}], [\textcolor{blue}{Fre09}]), we prove:

\emph{Theorem} \textbf{3.14} \emph{(Complementary collapse).}
Let \( \textbf{F} = (F_1, \ldots, F_n) \) \emph{be an orthogonality fan on \( P \) with \( F_1([\hat{0}]) \neq \hat{0}, \hat{1} \). There is a \( G \)-equivariant simple homotopy equivalence}
\[
|\overline{\mathcal{P}}|\xrightarrow{\simeq} \bigvee_{[y_0 < \cdots < y_r] \perp \textbf{F}} |\overline{\mathcal{P}}_{(\hat{0},y_0)}|^{\diamond} \wedge \Sigma|\overline{\mathcal{P}}_{(y_0,y_1)}|^{\diamond} \wedge \cdots \wedge \Sigma|\overline{\mathcal{P}}_{(y_{r-1},y_r)}|^{\diamond} \wedge |\overline{\mathcal{P}}_{(y_r,\hat{1})}|^{\diamond}.
\]


\end{document}
